\section{Comparison and Related work}
\label{sec:relatedwork}
%\subsection{Model-code synchronization}
%Several commercial and open-source tools \cite{EA, ibm_rational_2016, Magicdraw, umlgen} support round-trip engineering between UML models and code.
%Systematic reviews of some of these tools are available in \cite{Cutting2015}.
%Support for Java round-trip is prominent in most tools.
%Other languages such as C++ are only available in a few tools \cite{ibm_rational_2016, umlgen, Magicdraw}.
%Our methodological pattern does not focus on a particular programming language
%or a particular modeling language. Furthermore, the current implementation of our approach is dedicated
%to UML and C++, which is less supported by tools than Java.
%Usually these tools only support architectural elements on the model-side.
%The model cannot be used for full implementation and
%dependencies derived from
%method bodies are not considered during the round-trip. In our work, we assume that
%the model can be used for full implementation. Furthermore, our implementation analyzes
%C++ method bodies not only to reverse them to UML, but also to derive dependencies in the UML model.
%Some tools \cite{EA} only allow
%one of the artifacts, model or code, to be edited at a certain time.
%There is then no problem of synchronizing model and code since there are no concurrent changes, which limits their applicability.
%Finally, some tools \cite{umlgen} do not support a real incremental reverse or code generation;
%instead, they treat change (e.g. renaming) as deletion followed by addition.

% RTE restriction
\subsection{Round-trip engineering and co-evolution}
Some RTE techniques restrict the development artifact to avoid
synchronization problems.
Partial RTE and protected regions are introduced in \cite{czarnecki_multi-level_2006} to preserve code editions which cannot be propagated to models.
The mechanisms as discussed in Section \ref{sec:intro} are used for the separation of generated and non-generated code. 
EMF implements these techniques to allow users to embed user-code replacing the default generated code.
Yu et al. \cite{yu2012maintaining} propose to synchronize user-code and generated code through bidirectional transformation.
This latter does not, however, allow modifications in regions beyond the marked ones and thus prohibits the programmers from changing USMs' topology.
\ttt{Deep separation} proposed in \cite{zheng2012enhancing} overcomes the limitations of the separation mechanisms.
However, %as the authors state that, 
it does not allow to modify the system architecture at the code level as previously discussed.  
%Furthermore, as discussed in Section \ref{sec:intro}, there is still a possibility for the programmers to produce accidental changes, which cannot be reflected to the model.  




Regarding the co-evolution of component-based model and code, the authors in \cite{kramer2015change} use a super model, from which source code and component model can be derived as views for modifications.
However, code modifications made outside of their (limited) editor and violating their rules, e.g. only method bodies allow to be modified in code, are not updated to the super model. 
 
In \cite{Maro:2015:IGT:2814251.2814253}, the authors propose to integrate graphical and textual editors for UML profiles.
The goal is to allow developers to work graphically and textually, which is similar to our goal.
However, this approach is dependent on Eclipse technologies and embeds all modeling concepts to textual editors while XSeparation only introduces partially to allow programmers to be familiar with the syntax of their favorite programming language.

The authors in \cite{angyal_synchronizing_2008} propose to synchronize code with platform specific models by using a three-way merging approaches.
However, the approach only deals with syntactic synchronization while the code semantics such as state machines in source code is not taken into account.

%\paragraph{Viewpoint synchronization}
%
%% Viewpoint
%Both models and code can be considered simply as different viewpoints
%of the same system. Viewpoints enable the partitioning of the model of a system into several representations. 
%Synchronization between viewpoints is crucial to maintain their consistency.
%
%In \cite{eramo_change_2008} the authors improve the modeling of relationships and constraints between
%elements in different viewpoints in order to better guarantee the consistency of viewpoints.
%In \cite{goedicke_viewpoint-oriented_2000} the authors argue that inconsistencies will exist
%in systems developed with different actors, using different viewpoints. They suggest that tools
%must be able to tolerate inconsistencies. A distributed graph transformation is proposed to deal with the problem of formalizing the integration of multiple viewpoints in software development.
%Their work focuses on requirements engineering.
%In contrast, our approach targets specifically both model and code.
%Code is not usually considered in viewpoint synchronization because code is deemed to be too fine-grained.
%Furthermore, our approach does not require explicit modeling
%of relationships between model and code elements.

%\subsection{Model synchronization}

%Viewpoints synchronization is generalized by model synchronization for which there is
%an abundance of techniques presented in the literature. 
%Model synchronization aims
%to maintain consistency between a source model and a target model. 

\begin{comment}

Many model synchronization techniques require the explicit mapping of source model and target model.
The authors in \cite{Paesschen2005} propose an injective mapping of elements in the source model to
the target model. The mapping can be used for synchronization.
Techniques and technologies, such as Triple Graph Grammar (TGG) \cite{giese_incremental_2006},
and QVT-Relation \cite{Omg2008},
allow synchronization between source and target elements that have non-injective mappings.
The authors in \cite{Hermann2012} formalize TGG for synchronization of models that are concurrently edited.
All of these techniques require a mapping model to connect the source and target models
with typed traceability links, which need to be persisted in a model store \cite{Bergmann2011}.
This means that editing one model requires the presence of the other.
Our model-code synchronization approach does not require a mapping model and an artifact may be edited
independently of the presence of the other corresponding artifact.

Other techniques \cite{foster_combinators_2007} are based on bi-directional transformations, which comprise a forward transformation of
source to target model, and a backward transformation of target to source model.
Bi-directional transformations provides a novel mechanism for synchronization.
Indeed, some works \cite{Matsuda2015} derive a backward transformation based on forward
transformation.
However, such works do not offer any means to synchronize models that are concurrently edited.

A few approaches derive model synchronization from model transformation while allowing concurrent editions
of both source and target models.
In \cite{xiong_towards_2007} the authors propose to automatically derive
model synchronization of a source and a target model related by
%an ATL \cite{eclipse_foundation_eclipse_2016}
model transformation.
The synchronization is based on differentiating source and target model states
but reflectable addition of an element in the target model is not well handled according to \cite{xiong_towards_2007}.
Our approach is generic and does not depend on a specific technology. Furthermore, in our implementation
we propose to use modification events rather than state differences for incremental
transformations, necessary for synchronization.

\end{comment}

%\subsection{State machine code generation}

%\label{sec:relatedwork}
%Code generation from state machines has been received huge attention in automated software development and many approaches are proposed. 
%This section mentions some usual patterns and how our approach differs. 
%A systematic review of proposals is presented in \cite{Domnguez2012}. 
%Main approaches including switch/if, state table and state pattern are investigated.

%Switch/if is the most intuitive technique implementing a "flat" state machine. Two types of switch/if are supported. The first one uses a scalar variable representing the current active state \cite{Booch1998}. A method for each event processes the variable as a discriminator in switch/if statement. The second one uses a double nested switch/if and has two variables to represent the current active state and the event to be processed \cite{Douglass1999}. The latter are used as the discriminators of an outer switch statement to select between states and an inner one/if statement to decide how the event should be processed. The behavior code of the two types is put in one file or class. This practice makes code cumbersome, complex, difficult to read and less explicit when the number of states grows or the state machine is hierarchical. Furthermore, the first approach lets the code scatter in different places. Therefore, maintaining or modifying such code of complex systems is very difficult.

%Many techniques for code generation from USM are proposed.
%Switch/if %is the most intuitive technique implementing a "flat" state machine \cite{Booch1998}.
%and double dimensional state table approach \cite{Douglass1999} are intuitive and efficient techniques. 
%one dimension represents states and the other one all possible events. 
%These usually require a transformation from hierarchical to flatten ones and support only a small subset of USM concepts such as state and transition.
%However, the semantics of USMs containing pseudo states such as histories or join/fork are hardly preserved during the transformation. 
%The latter can be implemented by either
%using a scalar variable \cite{Booch1998} and a method for each event or using two variables as the current active state and the incoming event used as the discriminators of an outer switch statement to select between states and an inner one/if statement, respectively. 
  
%Each cell of the table is associated with a function pointer meaning that the state associated with a dimension index of the cell is triggered by the event associated with the other dimension. 
%The behavior code of these techniques is put in one file or class. This practice makes code cumbersome, complex, difficult to read and less explicit when the number of states grows or the state machine is hierarchical. 
%Therefore, maintaining or modifying such code of complex systems is very difficult. 
%Furthermore, these approaches requires every transition must be triggered by at least an event. This is obviously only applied to a small sub-set of USMs.  

%Some approaches such as state pattern \cite{Shalyto2006,Douglass1999} and its extension \cite{niaz_mapping_2004} produce object-oriented code by representing states as classes.
% to implement flat state machines. Each state is represented as a class and each event as a method. %The event is processed by a delegation from the context class to sub-states. 
%Separation of states in classes makes the code more readable and maintainable. %Unfortunately, this technique only supports flat state machines. 
%This pattern is extended in \cite{niaz_mapping_2004} to support hierarchical-concurrent USMs. 
%Recently, a double-dispatch (DD) pattern presented in \cite{spinke_object-oriented_2013} extends \cite{niaz_mapping_2004} to support maintainability. %by as a new technique to implement state machines. 
%representing states and events as classes, and transitions as methods. 
%However, these approaches do not generate code for full USM features and require much memory because of class explosion and uses dynamic memory allocation, which is not preferred in embedded systems \cite{spinke_object-oriented_2013}.
%However, the maintenance of the code generated by this approach is not trivial since it requires many small changes in different places. %This is impractical when dealing with large state machines. %Furthermore, similar to the state table, this approach also poses the requirement of having at least one event for transition.

%Tools, such as \cite{ibm_rhapsody, sparxsystems_enterprise_2014}, apply different patterns to generate code. 
%However, as mentioned in Section \ref{sec:intro}, true concurrency and some pseudo-states are not supported. 




%Our generation approach relies on and extends this approach. The latter profits the polymorphism of object-oriented languages. %provides some 1-1 mappings from state machines to object-oriented code and the implementation technique 
%is not dependent on a specific programming language. 
%However, DD does not deal with triggerless transitions and different event types supported by UML such as \ti{CallEvent}, \ti{TimeEvent} and \ti{SignalEvent}. Furthermore, DD is not a code generation approach but an approach to manually implementing state machines.




\subsection{Language engineering}
Text-based modeling languages (Textual ML) of UML such as PlantUML \cite{plantuml}, Umple \cite{lethbridge2010umplification}, and Earl Grey (EG) \cite{mazanec2012general} support UML class and State Machine elements.
However, these languages lack the explicit support for event types definitions used in UML.
Furthermore, they 
do not allow the programmers to reuse the existing syntax of object-oriented languages but redefine it in their own language and IDE, which is different with the idea of considering OOL as a hosting language and additional constructs as an internal language. 
By this way, Textual MLs need a tooling support such as compiler and editor, which require a lot of engineering tasks to develop. 
Contrarily, XSeparation only requires a light-weight compiler and enables using favorite IDEs. %while familiarity of these Textual MLs are questionable in \cite{mazanec2012general}. 
%XSeparation automatically synchronizes the code with the system model specified by UML.
%Hence, XSeparation profits all benefices of IDEs such as intelligent completion and easy to implement. 
Furthermore, XSeparation allows to use all specific OOL features such as pointers and references in C++ for program performance, which are not available in the Textual MLs.
%Followings list differences in comparison between RAOES and the TMLs.

\begin{comment}
\begin{itemize}[\footnotesize]
	\item RAOES adapts USM features to existing programming languages while Umple or TextUML does inversely, hence RAOES profits all benefices of IDEs such as intelligent completion and easy to implement. Furthermore, RAOES allows to use all specific C++ features such as function pointers for program efficiency, which are not available in the the TMLs.
	
	\item In RAOES, the programmers write and maintain the USM-based behavior part in the same class/file containing the active class.
	
	\item RAOES support full USM features.
	
	\item RAOES automatically synchronizes the code with the system model specified by UML.
	
	\item RAOES defines the state machine topology separately from the transition table and event definition.
\end{itemize}
\end{comment}


Similarly to Umple, \ti{mbeddr} \cite{voelter2012mbeddr} - an extensible C-based programming language introduces a new editor to mix high-level using state machines and components, and low-level code. 
\ti{mbeddr} is a code-centric approach and events are not UML-compliant whereas XSeparation conforms enables a model-code concurrent development, maintenance, and co-evolution of the artifacts.  

%The idea of adding more constructs for object-oriented languages to contain architecture information in XSeparation is similar to ArchJava \cite{aldrich2002archjava} and Archface \cite{ubayashi2010archface}.
%This latter utilizes Java annotations to preserve architecture intention in the source code.
%These approaches try to embed architectural information into the code, specifically Java while RAOES integrates the behavior represented by USMs into C++ to ease the RTE problem. 
ArchJava \cite{aldrich2002archjava} adds structural component concepts such as part and port to Java to support the co-evolution of architecture structure and Java implementation. 
The Archface \cite{ubayashi2010archface} contract description language supports components and connectors, between design and implementation using concepts such as \ttt{pointcut} in Aspect programming to reason about the design and code correctness.
The latter can also realized with XSeparation because we generated code can be used to reconstruct component models.
These approaches, however, do not provide the traceability between of the architecture behavior and code.
%The user-code and generated code are not separated as in XSeparation and \ttt{1.x-way mapping} to allow modifications made in both architecture and code.
They also do not have a synchronization mechanism in case of concurrent development.
Furthermore, the communication between two ports uses method calls of object-oriented languages instead of interfaces as in UML.
In addition, while ArchJava makes it not Java anymore and facilities of Java's IDEs such as auto-completion are not aware, XSeparation-generated code can use all of these IDE functionalities as discussed in \ref{sec:xseparationarchitecture}.

%FXU \cite{Pilitowski2007} is the most complete tool but generated code is heavily dependent on their own library and C\# is generated.



%In \cite{balz2009embedding}, the authors embeds behavior models into Java by representing, for example, states as classes, transitions as annotations, guards as methods.
%Although the programmers can modify the code following these patterns, the code size is large because of class explosion.

%\subsection{Code generation patterns and tools}
Tools such as IBM Rhapsody \cite{ibm_rhapsody}, Enterprise Architect \cite{EA}, Papyrus-RT \cite{possepapyrusrt}, and Sinelabore \cite{sinelabore} support only the structure RTE for UML class diagram concepts and code generation from UML State Machines.
Techniques for generating code from USM such as SWITCH/IF, state table \cite{Douglass1999} and state pattern \cite{Shalyto2006,niaz_mapping_2004} are proposed. 
A systematic review of code generation approaches is presented in \cite{Domnguez2012}.
%Switch/if %is the most intuitive technique implementing a "flat" state machine \cite{Booch1998}.
%and state table approach \cite{Douglass1999} are intuitive techniques. 
%Some approaches such as state pattern \cite{Shalyto2006,niaz_mapping_2004} produce object-oriented code.
% to implement flat state machines. Each state is represented as a class and each event as a method. %The event is processed by a delegation from the context class to sub-states. 
%Separation of states in classes makes the code more readable and maintainable. %Unfortunately, this technique only supports flat state machines. 
%This pattern is extended in \cite{niaz_mapping_2004} to support hierarchical-concurrent USMs. 
%Recently, a double-dispatch (DD) pattern presented in \cite{spinke_object-oriented_2013} extends \cite{niaz_mapping_2004} to support maintainability. %by as a new technique to implement state machines. 
%representing states and events as classes, and transitions as methods. 
However, only a subset of USM features is supported and generated code is not efficient, %of class explosion and uses dynamic memory allocation, 
which cannot be used for embedded systems \cite{spinke_object-oriented_2013}.
%Our source-to-source transformation combines SWITCH/IF the pattern in \cite{niaz_mapping_2004} to produce small footprint and preserve state hierarchy.
%Furthermore, 
RAOES offers code generation for all USM concepts. %including states, pseudo states, transitions, and events.
Therefore, users are free and flexible to create there USM conforming to UML without restrictions.