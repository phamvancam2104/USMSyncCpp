\section{Introduction}
\label{sec:intro}
%Unified Modeling Language (UML) State Machines (USMs) and their visualizations are suitable to model and design event-driven architecture-based embedded systems.
%The latter are usually resource-constrained.
%USMs can be used by the code generation technology in Model-Driven Engineering (MDE).
%The latter is considered as an efficient methodology to deal with system complexity.





%The so-called Model-Driven Engineering (MDE) approach relies on two paradigms, abstraction and automation \cite{Mussbacher2014}. It is recognized as very efficient for dealing with complexity of today's systems. 
%Abstraction is the ability to provide simplified and focused view of a system and requires adequate modeling language. 
%Unified Modeling Language (UML) \cite{Specification2007} is nowadays the most used, educated, and documented modeling language. 
%Even, if a graphical language such as Unified Modeling Language (UML) \cite{Specification2007} is not the silver bullet for all software related concerns, it provides hence better support than text-based solutions for some concerns such as architecture and logical behavior of application development. 
%Abstraction provides simplified and focused views of a system and requires adequate graphical modeling languages such as Unified Modeling Language (UML). %Even, if the latter is not the silver bullet for all software related concerns, 
%The latter provides better support than text-based solutions for some concerns such as architecture and logical behavior of application development. 

%Event-driven architecture is used for . 
%UML state machines (USMs) and their visual representations are efficient to modeling high level logic behaviors of event-driven embedded software architecture. % for designing memory-constrained 
%embedded systems \cite{Dunkels:2006:PSE:1182807.1182811}.
%UML class and composite structure diagrams prove to better capture the structure of such architecture \cite{possepapyrusrt,Ringert2013}.    
UML State Machines and composite structure are prove to efficiently capture and simplify the complexity in designing the behavior and structure, respectively, of event-driven and cyber-physical system architecture \cite{possepapyrusrt,Ringert2013}.
A number of code generation approaches have been proposed in the context of Model-Driven Engineering (MDE) \cite{kent2002model} to automate the process of translating the software architecture represented by these models into implementation 
\cite{possepapyrusrt, Douglass1999, ibm_rhapsody}.
%\cite{possepapyrusrt, Douglass1999,Shalyto2006, ibm_rhapsody}.  

Ideally, a full model-centric approach is preferred by MDE community due to its advantages \cite{Selic2012} such as complex project management, model-based system analysis, abstraction, and automation. 
The ultimate goal of MDE is to create executable models for generating full implementation code.
%However, in industrial practice, there is significant reticence \cite{Hutchinson:2011:MEP:1985793.1985882} to adopt it.


On the one hand, to generate full code, models must contain very detailed information. Nevertheless, current MDE tools and approaches are not sufficient to describe fine-grained behavior of the architecture using models. Therefore, to achieve the goal of complete code generation, some industrial MDE tools put fine-grained action code within the architecture model as blocks of text. MDE tools usually support limited editors for manual fine-grained coding tasks, which might produce errors during compilation of the generated code. When it happens, the developers tend to directly modify/correct the generated code for successful compilation because the code can be opened and easily modified within a familiar integrated development environment of the developers. Furthermore, some architecture-centric MDE approaches \cite{zheng2013classification} only produce skeleton code from architecture models. The generated code in this case must then be tailored by developers for fine-grained code. The modifications of generated code in these cases might violate the architecture correctness, which is not easy to detect due to the lack of a bidirectional traceability between the architecture and code \cite{ubayashi2010archface}.

On the other hand, in software evolution, continuous development and maintenance, the architects might change the architecture for new functionalities or requirements while the programmers might still tailor the current architecture or modify the code for various reasons such as code level optimization, bug fixing, refactoring. This results that the architecture model and code are concurrently modified. 




%However, on one hand, to generate full code, models must contain very detailed information. 
%Nevertheless, current MDE tools and approaches are not sufficient to describe fine-grained behavior of the architecture. 
%Some MDE tools such as IBM Rhapsody put fine-grained action code within the architecture model as blocks of text to generate full code. 
%However, it is not favorable because the architecture should only hold design decisions. 
%Hence, code generated from the current MDE tools must be tailored by programmers for fine-grained code. 
%Furthermore, it is frequent that programmers modify architecture information during implementation because of abstraction gap between architecture and implementation \cite{ubayashi2010archface}. 
%In these cases the architecture correctness might be violated, which is not easy to detect due to the lack of a bidirectional traceability between the architecture and code. 
%Some approaches \cite{kelly2008domain} prevent modifications in code through complete code generation.
%However, the latter only works for very highly specialized domains and specific architecture styles \cite{zheng2012enhancing}.

%On one hand, it is frequent that programmers modify architecture information during implementation because of abstraction gap between architecture and implementation. %It is not easy to preserve the architecture correctness in implementation. 
%Current industrial MDE tools such as IBM Rhapsody put fine-grained behaviors/computational algorithms directly in the architecture model as blocks of text to generate full code. 
%However, it is %difficult for system analysis and 
%not favorable because the architecture should only hold design decisions. 
%Hence, code generated from the current MDE tools must be tailored by programmers for fine-grained code.
%In this case, the architecture correctness might be violated, which is not easy to detect due to the lack of a bidirectional traceability between the architecture and code \cite{ubayashi2010archface}.
%Some approaches \cite{kelly2008domain} prevent modifications in code through complete code generation.
%However, the latter only works for very highly specialized domains and specific architecture styles \cite{zheng2012enhancing}.

%On the other hand, in software evolution, continuous development and maintenance, the architects might change the architecture for new functionalities or requirements while the programmers might still implement the current architecture or modify the code for various reasons such as code level optimization, bug fixing, refactoring. 

\begin{comment}
In MDE tools, the regeneration of code from the changed architecture would overwrite the modifications made by programmers in code. 
Some tools such as Eclipse Modeling Framework (EMF) \cite{steinberg2008emf} separate code areas, which could be modified by the programmers, to preserve the code changes by using some specialized comments such as \ttt{@generated NOT}.
However, current separation mechanisms require the programmers to be very highly discipline.
Furthermore, even so, accidental changes are still possible \cite{zheng2012enhancing}.
The \ttt{1.x-way architecture mapping deep separation} approach \cite{zheng2012enhancing} overcome the limitations of these separation mechanisms by generating \ttt{architecture-prescribed code} in a class separating from user-code written in an other class.
However, deep separation does not allow to modify the architecture at the code level.
\end{comment}

The modifications made in model and code raise the consistency and synchronization problem.  If the latter is not solved, modifications in the code are not reflected to the model. As a result, the model does not reflect the actual running system, which even worse entails that model-based activities such as architecture and behavior analysis, or testing are obsolete, hence many of the advantages of MDE would disappear.
%The modifications made in model and code raise the consistency and synchronization problem \cite{zheng2013classification}.
%If the latter is not solved, modifications in the code are not reflected to the model. 
%As a result, the model does not reflect the actual running system, which even worse entails that model-based activities such as architecture and behavior analysis, or testing are obsolete, hence many of the advantages of MDE would disappear.
In order to solve this problem, which hinders the adoption of MDE in practice, it is necessary to have a code generation process, which establishes a way to trace code modifications back to the model.

In this paper, we propose \tb{XSeparation} - an extreme separation code generation and synchronization approach. 
%, which consists of structure and behavior models
 and code.
The core idea of XSeparation is to leverage an object-oriented language by engineering additional constructs to it in order to enable the bidirectional traceability between architecture model.
XSeparation provides an adequate support for programmers to control both structure and behavior of a component at the code level. %by making changes in generated architecture-prescribed and state machine-based behavior-prescribed code, respectively.
%XSeparation increases the bidirectional traceability to keep the model and the code consistent. 
%Our goal is to synchronize the system architecture described by UML class and component diagrams and the behavior by UML State Machines with object-oriented code such as C++ and Java.
%These architecture and behavior models, briefly as the design model, are then used for generating code (implementation).
%The code can be modified by programmers while in the meantime the architecture model might be modified by software architects.
%The concurrent modifications raise the problem of consistency between the architecture model and implementation code.

%Synchronization of concurrent modifications made in the architecture model and code is considered as a hard problem because of the abstraction gap between the architecture and the implementation (code). 
%This gap makes the bidirectional traceability, which allows to reflect modifications made in one artifact to the other, between model and code hard, even impossible.


From a research perspective, this study aims to improve flexibility in MDE to allow the architecture model and the generated code to co-evolve while keeping these two consistent \cite{yu2012maintaining}. 
Furthermore, R. N. Taylor et al. \cite{Taylor:2007:SDA:1253532.1254721} pointed out an important research direction, in which key design decisions may be made in implementation (code) and evolution of architecture must be seamlessly propagated to the code [5]. 
This implies the fluid moving from architecture to code and vice versa. Additionally, synchronization of model and code is also considered as an important need by the MODELS community \cite{van2008challenges}.

For industry, one of the reasons that impede the adoption of MDE is the perceived gap between diagram-based modeling languages and textual languages. On one hand, programmers prefer to use the more familiar combination of a programming language and Integrated Development Environment (IDE). On the other hand, software architects, working at higher levels of abstraction, tend to favor the use of modeling languages for describing the system architecture.


%Our work is motivated by both industry and research.
%From the latter, MDE, for flexibility, allows the architecture model and the generated code to evolve concurrently \cite{yu2012maintaining}.
%Furthermore, R. N. Taylor et al. \cite{Taylor:2007:SDA:1253532.1254721} pointed out an important research direction, in which key design decisions may be made in implementation (code) and evolution of architecture must be seamlessly propagated to the code. 
%This implies the fluid moving from architecture to code and vice versa. 
%Additionally, synchronization of model and code is also considered as an important need by the MODELS community \cite{van2008challenges}.

%Furthermore, as addressed in \cite{zheng2012enhancing}, bidirectional mapping (two-way mapping) between the design model (architecture + behavior model) and the code is the most promising among others such as \ttt{correct-by-detection} and \ttt{one way mapping}.
%This is because the bidirectional mapping provides concurrent modifications made in the design model and the code to foster for software architects and programmers collaboration.

%For industry, one of the reasons that impede the adoption of MDE is the perceived gap between diagram-based modeling languages and textual languages.
%On one hand, programmers prefer to use the more familiar combination of a programming language and Integrated Development Environment (IDE). 
%Text editors
%like Emacs and Vim are also favored by some programmers in
%the embedded Linux community. 
%On the other hand, software architects, working at higher levels of abstraction, tend to favor the use of modeling languages for describing the system architecture.

%In order to provide better support for industry to raise the adoption of MDE synchronization, our goal is to automate the mode-code synchronization in order to maximize the effectiveness of both modeling and programming world \cite{zheng2013classification}.
	
%Software architects, working at a high level abstract, prefer using graphic-based modeling languages for architecture and logic behavior (via UML State Machine) description.
	
%Programmers favor the use of text-based programming languages with their preferred integrated development environment for fine-grained statements and computational algorithms.



%On one hand, programmers prefer to use the more familiar programming language. 
%On the other hand, software architects, working at higher levels of abstraction, favor the use of models, and therefore prefer graphical languages for describing the system architecture and the high level logic behavior \cite{Hutchinson:2011:MEP:1985793.1985882,Hutchinson:2011:EAM:1985793.1985858}.
%Furthermore, a common practice in industry is to use improper languages, C++/Java e.g., to define fine-grained actions within models.
%Due to several reasons such as bug fixing or code level optimization, code is usually refined/modified after generation.


%The code modified by programmers and the model are then inconsistent. 
%This is considered as the well-known Round-trip engineering (RTE) \cite{Hettel2008} problem in MDE.
%is proposed to synchronize different software artifacts, model and code in this case \cite{Sendall}. 
%RTE enables actors (software architect and programmers) to freely move between different representations and stay efficient with their favorite working environment. 
%In other words, RTE enables both model and code to be considered as development artifact. 

%\subsection{Problem definition and challenges}
\label{subsec:problemdefinition}
The system architecture is described by UML class and component diagrams and the behavior by UML State Machines.
These architecture and behavior models, briefly as the design model, are then used for generating code (implementation).
The code can be modified by programmers while in the meantime the architecture model might be modified by software architects.
The concurrent modifications raise the problem of consistency between the architecture model and implementation code.

Synchronization of concurrent modifications made in the architecture model and code is considered as a hard problem because of the abstraction gap between the architecture and the implementation (code). 
This gap makes the bidirectional traceability, which allows to reflect modifications made in one artifact to the other, between model and code hard, even impossible.


%\subsection{Why is it a problem? Why are there concurrent modifications?}
\label{subsec:reasons}

\noindent
\circled{1} \tb{Architecture and implementation abstraction gap}
\begin{itemize}
	\item It is frequently that programmers modify architecture information during implementation. %It is not easy to preserve the architecture correctness in implementation. 
	Current industrial MDE settings put fine-grained behaviors/computational algorithms directly in the architecture model to generate full code. 
	However, it is %difficult for system analysis and 
	not favorable because the architecture should only hold design decisions. 
	Hence, code generated from the current MDE tools must be tailored by programmers for fine-grained code.
	In this case, the architecture correctness might be violated, which is not easy to detect due to the lack of a bidirectional traceability between the architecture and code \cite{ubayashi2010archface}.
	
	%\item A common practice in industry is to use improper languages, C++/Java e.g., to define fine-grained actions within models. 
	
	
	\item In software evolution, continuous development and maintenance, the architects might change the architecture for new functionalities or requirements while the programmers might still implement the current architecture or modify the code for various reasons such as code level optimization, bug fixing, refactoring. 
	In MDE tools, the regeneration of code from the changed architecture would overwrite the modifications made by programmers in code. 
	Some tools such as Eclipse Modeling Framework (EMF) \cite{steinberg2008emf} separate code areas, which could be modified by the programmers, to preserve the code changes by using some specialized comments such as \ttt{@generated NOT}.
	However, current separation mechanisms require the programmers to be highly discipline.
	Furthermore, even so, accidental changes are still possible \cite{zheng2012enhancing}.
	The \ttt{1.x-way architecture mapping deep separation} approach \cite{zheng2012enhancing} overcome the limitations of these separation mechanisms by generating \ttt{architecture-prescribed code} in a class separating from user-code written in an other class.
	However, deep separation does not allow to modify the architecture at the code level.
	%However, code generated by these tools produce laborious comments which make the code ugly and the programmers feel hard to read and modify. 
\end{itemize} 


%However, the violation in the code is not esy to detect because there is no traceability between architecture and implementation => we need a bidirectional traceability between architecture and implementation. Unfortunately, current MDD tools are insufficient to realize this kind of bidirectional traceability.


%\vskip 0.03in
%\noindent
%\circled{2} \tb{Continuous and concurrent development and maintenance:}
%Why code modifications?
%Code level optimization, bug fixing, refactoring (renaming, i.e.)

%Architecture is not realistic in programming

%Programmers do not only modify method bodies, but also structure, methods to adopt well-known programming paradigm

%Rarely, the programmers do not change anything in architecture, if changes onccur, they have to be propagated back to the architecture




\vskip 0.03in
\noindent
\circled{2}\tb{Architecture and programmer perception}
\begin{itemize}
	\item Working with code is easier for programmers in solving computational/algorithmic problems than with models.
	
	\item Software architects, working at a high level abstract, prefer using graphic-based modeling languages for architecture and logic behavior (via UML State Machine) description.
	
	\item Programmers favor the use of text-based programming languages with their preferred integrated development environment for fine-grained statements and computational algorithms.
\end{itemize}



%\subsection{Contribution}
\label{subsec:contribution}
In this paper, we propose \tb{XSeparation} - an extreme separation code generation approach, synchronization, and compilation, which enable the bidirectional traceability between software model, which consists of architecture and behavior models, and code.
XSeparation lifts the \ttt{deep separation} approach to an extreme level of separation.
XSeparation provides adequate support for programmers to control both architecture and state machine-based behavior of a component at the code level by making changes in generated architecture-prescribed and state machine-based behavior-prescribed code, respectively.
The changes are then reflected to the model through the bidirectional traceability to keep the model and the code consistent. 


 

%The surveys described in \cite{Hutchinson:2011:MEP:1985793.1985882} and \cite{Hutchinson:2011:EAM:1985793.12985858} polled MDE practitioners. 
%It notes that 70\% of the respondents primarily work with models, but still require manually-written code to be integrated.
%Furthermore, 35\% of the respondents answered that they spend a lot of time and effort synchronizing model and code.

%However, on the one hand, maintaining code generated from existing approaches is non-trivial. On the other hand our observation is that it is very difficult to come up with formalizations that yield such elegant code generation solutions \cite{6032552}. In other words, generated code must be manually modified to build fully operational applications. 
%On one hand there are traditional developers who prefer to implement the system by writing code, while on the other hand there are developers who prefer to use entirely models for the design and implementation of the system. 



%After code modifications, round-trip engineering (RTE) is needed to make the model and code consistent, which is a critical aspect to meet quality and performance constraint required from project managers today. 

%Approaches proposed for RTE are categorized as \ti{structure} and \ti{behavior} RTE.
\ti{structure} RTE refers to synchronization of structural concepts such as those available from class diagrams and code, and is supported by industrial tools such as IBM Rhapsody \cite{ibm_rhapsody} and Enterprise Architect \cite{sparxsystems_enterprise_2014}.
Some approaches such as \cite{langhammer2013co, kramer2015change} allow the co-evolution of component-based diagram elements and code.

The \ti{behavior} RTE is usually supported very limitedly. 
This is because there is no trivial mapping from behavior model such as USM and code.
Consequently, it is difficult to reflect behavior code changes to the original model.
Some approaches support a partial behavior RTE, which often allows programmers to partially modify behavioral code in limited areas by separating the \ttt{generated} and \ttt{non-generated code} \cite{ibm_rhapsody,steinberg2008emf} using some specialized comments such as \ttt{@generated NOT}.
Approaches and tools in this category use an incremental code generation, which preserves the user-code changes in the areas marked as non-generated.
However, "\ti{current separation mechanisms require the programmers are highly discipline.
Furthermore, even so, accidental changes are still possible}" \cite{zheng2012enhancing}.


In this paper, we tackle the problem of synchronization between model, which includes both architectural and behavioral elements, and \ttt{C++} code, which meets resource-constrained requirements for developing event-driven embedded systems.
Specifically, the system architecture is specified via UML class, component, and composite structure diagrams, and the behavior via USMs.
%Component and composite structure diagrams for architecture description will be included in future work.
To support the architects and the programmers at the modeling and programming level, respectively, our goal is to allow the synchronization of code and USMs with full features.
Generated code should be efficient (small in size and fast in event processing speed) to be fit into resource-constrained systems. 

Our proposed technique \tf{RAOES} - Round-trip engineering for Active Objects-based Embedded System is inspired by \ttt{ArchJava} \cite{aldrich2002archjava} and \ttt{Archface} \cite{ubayashi2010archface} whose goal is to allow the co-evolution of architecture and implementation in Java by introducing additional constructs to Java.
Our approach adds  a subset of UML-based constructs to C++ in order to connect it to USMs.
Instead of directly generating C++ code from models as the existing tools, RAOES produces a C++ front-end code, which contains our added constructs.
The programmers are free to modify not only the high level logic behavior described by USMs but also the user code by making changes to the C++ front-end code.

The introduction of the front-end is similar to \ttt{MSM} \cite{MSM} and \ttt{EUML} \cite{EUML}.
However, these front-ends use a lot of C++ templates, which make the code difficult to write and understand.
Furthermore, they support only a limited subset of USM, especially events defined by UML are not correctly supported.

In RAOES, the C++ front-end is merged into and written in the usual C++ code.
The front-end is then used for generating a back-end code, which is actually used for compilation to binary files.
Furthermore, using our strategy defined in this paper, the front-end code is also synchronized with the model when there are concurrent modifications.

%Unfortunately, current industrial tools such as for instance Enterprise Architect \cite{sparxsystems_enterprise_2014} and IBM Rhapsody\cite{ibm_rhapsody} only support structural concepts for RTE such as those available from class diagrams and code. Compared to RTE of class diagrams and code, RTE of USMs and code is non-trivial. It requires a semantical analysis of the source code, code pattern detection and mapping patterns into USM elements. 
%This is a hard task, since mainstream programming languages such as C++ and JAVA do not have a trivial mapping between USM elements and source code statements.

%For software development, one may wonder whether this RTE is doable. That is, why do the industrial tools not support the propagation of source code modifications back to original state machines? Several possible reasons to this lack are (1) the gap between USMs and code, (2) not every source code modification can be reverse engineered back to the original model, and (3) the penalty of using transformation patterns facilitating the reverse engineering that may not be the most efficient (e.g. a slightly larger memory overhead). 
%in the mind of these tools' vendors, users always make changes to models rather than to code. Generated code, in these tools, is therefore not supposed to be changed directly.  

%In this paper, we address the RTE of UML State Machine diagrams and its related generated code. We propose a RTE approach consisting of a forward process which generates code by using transformation patterns, and a backward process which is based on code pattern detection to update the original state machine model from the modified generated code. From the proposed approach, we implemented a prototype and conducted several experiments on different aspects of the round-trip engineering to verify the proposed approach. 



%Model-driven engineering (MDE) is a development methodology aiming to increase software productivity and quality by allowing different stakeholders to contribute to the system description \cite{Mussbacher2014}. MDE considers models as first-class artifacts and generates code from higher abstraction level models. Recent survey \cite{1030} has revealed that industries are gaining the adoption of code generation into software development life-cycle. Although many tools and research prototypes can generate executable code from models, generated code could be manually modified by programmers, e.g. skeleton code generated from UML \cite{Specification2007} class diagrams. Models and the generated code are therefore out of synchronization. Round-trip engineering \cite{Aßmann200333, Hettel2008, E-ESE-120044648} (RTE) is proposed to keep the artifacts synchronized.

%RTE supports synchronizing different software artifacts, model and code in this case, and thus enabling actors (software architect and programmers) to freely move between different representations \cite{Sendall}. Tools such as for instance Enterprise Architect \cite{sparxsystems_enterprise_2014}, Visual Paradigm \cite{visual}, and AndroMDA \cite{_andromda_} provide RTE but most of them are only applicable for system structure models such as class diagrams.  

%This study addresses the RTE of UML State Machine (SM) and object-oriented programming languages such as C++ and JAVA. SM is widely used in practice for modeling the behavior of complex systems, notably reactive, real-time embedded systems. There are several approaches to generating source code from state machines or state charts such as nested switch/if statements \cite{Booch1998}, state-event-table \cite{Douglass1999, Duby2001}, and state pattern \cite{Allegrini2002,Shalyto2006,Douglass1999}. Unfortunately, the generated code from these approaches is very difficult for programmers to maintain without an appropriate supporting tool. RTE is impossible in these approaches even with very small changes such as changing transition targets or actions made to code. The reason behind this impossibility is that, in mainstream programming languages such as C++, JAVA, (1) there are not equivalents between SMs and source code statements and (2) the code generation pattern of these approaches has not been chosen with RTE in mind.

%This paper addresses the RTE of UML state-machines and object-oriented programming languages such as C++ and JAVA. The forward  engineering of the approach takes as input a state-machine and executes two transformations. The first is UML to UML by utilizing several transformation patterns such as the double-dispatch approach presented in \cite{spinke_object-oriented_2013} and the second is a generation of code from the transformed UML. Traceability information is stored, during the transformations. In the backward direction, a verification is executed by the code pattern detection to verify the correctness of the code before the backward process taking as input the modified generated code, the UML classes, the original state-machine and mapping information together merges changes from code to the state-machine. We implemented a prototype supporting RTE of state-machine and C++ code, and conducted several experiments on different aspects of the RTE to verify the proposed approach. To the best of our knowledge, our implementation is the first tool supporting RTE of SM and code. 
%The prototype also improves the collaboration between MDE developers and traditional programmers in developing reactive complex embedded systems.

%This paper addresses the RTE of USMs and object-oriented programming languages such as C++ and JAVA. The main idea is to utilize transformation patterns from USMs to source code that aggregates code segments associated with a USM element into source code methods/classes rather than scatters these segments in different places. Therefore, the reverse direction of the RTE can easily statically analyze the generated code by using code pattern detection and maps the code segments back to USM elements. Specifically, in the forward direction, we extend the double dispatch pattern presented in \cite{spinke_object-oriented_2013}. Traceability information is stored during the transformations. We implemented a prototype supporting RTE of state-machine and C++ code, and conducted several experiments on different aspects of the RTE to verify the proposed approach. To the best of our knowledge, our implementation is the first tool supporting RTE of SM and code. 

To sum up, our contribution is as followings:
\begin{itemize}
	\item RAOES: A full round-trip engineering approach for developing event-driven systems using UML State Machines and C++.
	\item The implementation of RAOES based on the Eclipse Modeling Framework (EMF) and the Papyrus tool.
	\item Experimental evaluations by experimenting with RAOES and a case study simulation.
	%\begin{itemize}
		%\item An automatic evaluation of the proposed RTE approach with the prototype.% including 300 random generated SM models containing 80 states, more than 230 transitions, more than 250 actions and around 180 events for each.
		%\item A complexity analysis of the approach and performance evaluation.
		%\item A comparison and collaboration of two software development practices including working at the model level and at the code level.
		%\item A lightweight evaluation of the semantic conformance of the runtime execution of generated code.
	%\end{itemize}
\end{itemize}

%Even though this presented work is specific to C++ and embedded systems, the general idea can be applied to other object-oriented programming languages such as Java and to other domains.

Our contribution is summarized as followings:

\begin{itemize}
	\item XSeparation - code generator, mode-code synchronizer, and compiler for enabling a fluid moving between architecture and implementation.
	
	\item Evaluations of XSeparation based on a case study.
\end{itemize}

The remainder of this paper is organized as follows: Section \ref{sec:butshell} presents the overview of XSeparation code generation. 
Section \ref{sec:xseparationarchitecture} and \ref{sec:xseparationbehavior} describe how XSeparation is applied to code generation from architecture structure and behavior, respectively, for enabling bidirectional traceability between model and code. 
%The syntax of RAOES's front-end is detailed in Section \ref{sec:syntax}.
Section \ref{sec:collaboration} shows how to synchronize concurrent modifications architecture model and XSeparation-generated code.
The development of a case study is presented in Section \ref{sec:evaluation} to evaluate XSeparation. 
Section \ref{sec:relatedwork} shows related work. 
The conclusion and future work are presented in Section \ref{sec:conclusion}.

