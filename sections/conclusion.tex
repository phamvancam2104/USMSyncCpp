\section{Conclusion}
\label{sec:conclusion}
We have presented RAOES-a round-trip engineering approach for effective collaboration between software architects and programmers in developing and maintaining reactive embedded systems using UML State Machines for describing behaviors. 
%The design for concurrency of generated code is based on multi-thread.
%The code generation pattern set extends the IF-ELSE/SWITCH patterns and the state pattern extension. 
%The hierarchy of USM is kept by our simple state structure.
RAOES introduces a C++ front-end code lying between models and actual executable code.
RAOES proposes a framework for synchronizing the models and the front-end code, and generating executable code from the front-end.
RAOES is implemented as an extension of the Papyrus modeling tool.

We evaluated RAOES by conducting experiments on the round-trip engineering correctness, the semantic-conformance and efficiency of generated code.
300 random models are tested for the RTE correctness.
The conformance is tested under PSSM that 62/66 tests passed.
For efficiency of code, RAOES produces code that runs faster in event processing time and is smaller in executable size than those of other approaches (in the paper scope).

For the moment, RAOES supports the co-evolution of UML class and state machine diagrams and code.
In the future, we will integrate component-based concepts into C++ to make the front-end more powerful and effective.
Our wish is to embed component-connector and interaction component information into C++ to stay synchronized with composite structure diagrams in UML and elaborate the application range such as distributed embedded systems.

However, code produced by PSM consumes slightly more memory than the others.
Furthermore, some PSSM tests are failed.
Therefore, as a future work, we will fix these issues by making multi-thread part of generated code more concise.  
 