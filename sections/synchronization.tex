\section{Synchronization}
\label{sec:collaboration}

\begin{comment}
%\subsection{Synchronization in case of concurrent modifications}
In our previous work \cite{foster2016}, a model-code synchronization methodological pattern in case of concurrent modifications is proposed.
This section extends the pattern to maximize automation and minimize user intervention in synchronization of architecture model and intermediate code.
In order to keep the paper focused, we abstractly present the synchronization as below.

Given a modified model \tb{ModM}, its previous/ancestor version \tb{PrevM}, and the modified intermediate code \tb{ModCode}, our synchronization consists of five steps as in Fig. \ref{fig:syncextended}. 

\vskip 0.1cm
\noindent
\tb{Step 1}:
We transform the modified intermediate code \tb{ModCode} into a corresponding model \tb{ReversedM} using the bidirectional traceability established in the previous sections.

\vskip 0.1cm
\noindent
\tb{Step 2}:
We compare \tb{ReversedM} and \tb{PrevM} to find differences \tb{Diff} between the two models.
Currently, we rely on EMFCompare.
Diff??


\vskip 0.1cm
\noindent
\tb{Step 3}:
From the differences set \tb{Diff}, we create a new model \tb{Model with Code Modifications} by merging \tb{Diff} into \tb{PrevM}.
After this step, \tb{Model with Code Modifications} is a successor of its ancestor model \tb{PrevM}.
The successor does not only reflect modifications made in the intermediate code but also the model elements which are not used for code generation but other model activities such as model-based analysis for safety and security (because we allow models to contain fine-grained code, full implementation code can be derived from the models. Therefore, the model always contains more information than the code. Preservation of model information is needed).

\vskip 0.1cm
\noindent
\tb{Step 4}:
At this point, both \tb{ModM} and the successor are considered as models diverged from the common ancestor \tb{PrevM}.
\tb{ModM} is created by model modifications made to \tb{PrevM} and the successor by code modifications made to the associated intermediate code of \tb{PrevM}.
We therefore use a three-way merger.

\vskip 0.1cm
\noindent
\tb{Step 5}:

\begin{figure}
	\centering
	\includegraphics[clip, trim=0cm 12.7cm 7.8cm 0cm, width=\columnwidth]{figures/syncextended.pdf}
	\caption{Synchronization process} 
	\label{fig:syncextended}
\end{figure}

\end{comment}

The pattern especially requires the availability of several use-cases as followings:

\begin{itemize}[\footnotesize]
	\itemsep0em
	\item \tb{Batch code generation}: generates and overwrites any existing code from model.
	\item \tb{Incremental code generation}: updates the code by propagating changes from the model to the code.
	\item \tb{Batch reverse engineering}: creates and overwrites any existing model from code.
	\item \tb{Incremental reverse engineering}: updates the model by propagating changes from the code to the model.
\end{itemize}


The batch code generation and reverse engineering are straightforwardly supported by using the bidirectional traceability between the architecture model and code created by XSeparation. The incremental code generation (ICG) and incremental reverse engineering (IRE) needs a classification and management of modifications made in the model and code.

\vskip 0.2cm
\noindent
\tb{Model modification classification and management:}
Table \ref{table:modelchangeclassification} shows our classification and management of actions for propagating modifications, detected by a model listener, in model to code.
%The modifications can be detected by a model listener.
We make a distinction between structural and behavioral modifications, which result in creating/removing/regenerating the corresponding code part.
Although only add/remove/update are detected, the moving of a model element can be combined as a removal following by an addition.  
Some particular modifications requires the corresponding actions to respect and preserve the user code.
For example, if the \ttt{sendDataToFifo} method in Listing \ref{lst:producerinteraction} is renamed to \ttt{sendDataFromProducerToFifo} at the model level, the corresponding action consists of several steps: (1) identify the method, at the code level, associated with the operation, at the model level, using the old name \ttt{sendDataToFifo}, which is recorded by the model listener;
(2) rename the method to \ttt{sendDataFromProducerToFifo} while keeping its parameters and body intact. 



\begin{comment}
\vskip 0.2cm
\noindent
\tb{Code modification classification and management:}
Modifications types in code are similar to that of model consisting of structural and behavioral modifications.
However, we do not support the removal of classes in code because this kind of modification causes conflicts and syntax errors, which require the code to be manually re-factored for reconciliation. 
We believe that an automatic mechanism brought by the regeneration of code caused by the removal can better handle the refactoring.
If the FIFO class is removed in the code, for example, the \ttt{fifo} in \ttt{System} in Listing \ref{lst:architectureprescribed} must be retyped or removed.
If not, a compilation error is raised.
% Please add the following required packages to your document preamble:
% \usepackage{multirow}
\end{comment}
\begin{table*}[]
	\scriptsize
	\centering
	\caption{Model change classification and management}
	\label{table:modelchangeclassification}
	\begin{tabular}{|l|p{3cm}|l|p{9.5cm}|}
		\hline
		\multicolumn{2}{|c|}{Element type}                              & \multicolumn{1}{c|}{Modification type} & \multicolumn{1}{c|}{Action}                                                                              \\ \hline
		\multirow{3}{*}{Structure} & Part/Port/Connector                & Add/Remove/Update                      & Regenerate Component structure-prescribed code                                                           \\ \cline{2-4} 
		& Class/Component/Interface          & Add/Remove/Update                      & Create/Remove/Update the corresponding code                                                              \\ \cline{2-4} 
		& Property                           & Add/Remove/Update                      & Create/Remove/Regenerate the corresponding class attribute                                               \\ \hline
		\multirow{4}{*}{Behavior}  & Operation                          & Add/Remove/Update                      & Create/Remove/Regenerate the corresponding method with keeping its method body                           \\ \cline{2-4} 
		& \multirow{3}{*}{UML State Machine} & Create                                 & Generate Behavior-prescribed code and State machine action code                                          \\ \cline{3-4} 
		&                                    & Remove                                 & Remove Behavior-prescribed code and State machine action code from the corresponding component           \\ \cline{3-4} 
		&                                    & Update                                 & Regenerate Behavior-prescribed code and State machine action code with respect to the existing user-code \\
		\hline
	\end{tabular}
\end{table*}

%Using these use-cases and the bidirectional traceability facilitated by XSeparation during code generation, the concurrent modifications in model and code can be synchronized by our previous methodological pattern.

After synchronization, to be executable, the XSeparation-generated code needs to be compiled.
In the next section, we will present the architecture of our XSeparation compiler for compilation. 

\begin{comment}
\begin{table}[]
	\centering
	\caption{My caption}
	\label{my-label}
	\begin{tabular}{lllll}
		UML                  & XGC                      &  & OO                             & C++                 \\ \cline{1-2} \cline{4-5} 
		Class component      & Class                    &  & Class                          & Class               \\ \cline{1-2} \cline{4-5} 
		Part                 & Part                     &  & Composition attribute          & Attribute           \\ \cline{1-2} \cline{4-5} 
		Port  (data/control) & Port                     &  & Attribute                      & Reference Attribute \\ \cline{1-2} \cline{4-5} 
		Many ports           & Multiple-port            &  & Multiple interface realization & --                  \\ \cline{1-2} \cline{4-5} 
		Connector            & Binding (static+dynamic) &  & --                             & Methods             \\ \cline{1-2} \cline{4-5} 
		Interface            & Class/Interface          &  & Interface                      & Class               \\ \cline{1-2} \cline{4-5} 
		Signal               & Class                    &  & Class                          & Class/Struct        \\ \cline{1-2} \cline{4-5} 
		State machine        & state\_machine           &  & --                             & --                  \\ \cline{1-2} \cline{4-5} 
		State                & state                    &  & --                             & --                  \\ \cline{1-2} \cline{4-5} 
		Region               & region                   &  & --                             & --                  \\ \cline{1-2} \cline{4-5} 
		CallEvent            & call\_event              &  & --                             & --                  \\ \cline{1-2} \cline{4-5} 
		TimeEvent            & time\_event              &  & --                             & --                  \\ \cline{1-2} \cline{4-5} 
		ChangeEvent          & change\_event            &  & --                             & --                  \\ \cline{1-2} \cline{4-5} 
		SignalEvent          & signal\_event            &  & --                             & --                  \\ \cline{1-2} \cline{4-5} 
		Any                  & any                      &  & --                             & --                  \\ \cline{1-2} \cline{4-5} 
		Pseudo state         & pseudo\_state            &  & --                             & --                  \\ \cline{1-2} \cline{4-5} 
		Action/Effect        & Method                   &  & Method                         & Method             
	\end{tabular}
\end{table}
\end{comment}

\begin{table}[]
	\centering
	\caption{Mapping between UML and Examples of Extended Language (1)}
	\label{table:mapping}
	\begin{tabular}{p{1.6cm}p{2.5cm}p{3.7cm}}
		UML                                                                      & Extended Language                                                                                & Code example in Fig. \ref{fig:approachexample}                                                                               \\ \hline
		\begin{tabular}[c]{@{}l@{}}Port requiring \\ an interface \ti{I}\end{tabular} & \begin{tabular}[c]{@{}l@{}}Attribute typed \\ by \ti{RequiredPort}\textless I\textgreater\end{tabular} & \begin{tabular}[c]{@{}l@{}}Ports \ti{pPush} and \ti{pPull} at lines\\ 19 and 22\end{tabular}         \\ \hline
		\begin{tabular}[c]{@{}l@{}}Port providing \\ an interface \ti{I}\end{tabular} & \begin{tabular}[c]{@{}l@{}}Attribute typed\\ by \ti{ProvidedPort}\textless I\textgreater\end{tabular}  & \begin{tabular}[c]{@{}l@{}}Ports \ti{pPush} and \ti{pPull} at \\ lines 26-27\end{tabular}            \\ \hline
		\begin{tabular}[c]{@{}l@{}}Bidirectional \\ port providing \\ \ti{R} and \ti{P} \end{tabular} & \ti{BidirectionalPort<R,P>} & Not shown in this paper
		\\ \hline
		Connector                                                                & Binding                                                                                        & Lines 7-8                                                                                  \\ \hline
		State Machine                                                            & \ti{StateMachine}                                                                                     & \begin{tabular}[c]{@{}l@{}}The FIFO state machine at \\ lines 31-51\end{tabular}           \\ \hline
		State                                                                    & \ti{State/InitialState}                                                                               & \begin{tabular}[c]{@{}l@{}}State \ti{SignalChecking} at \\ lines 33-36\end{tabular}             \\ \hline
		Region                                                                   & \ti{Region}                                                                                           & Not shown in this paper                                                                                 \\ \hline
		Pseudo state                                                             & \begin{tabular}[c]{@{}l@{}}Attribute typed \\ by pseudo type\end{tabular}                        & \begin{tabular}[c]{@{}l@{}}The \ti{dataChoice} pseudo state \\ at line 42\end{tabular}          \\ \hline
		Action/Effect                                                            & Method                                                                                           & Methods at lines 52-57       \\ \hline
		Transitions                                                           & Transition table                                                                                           & Transition table at lines 44-50	\\ \hline
		Event                                                            & Event                                                                                           & The call event at line 43       \\ \hline           
		Deferred Event	& \begin{tabular}[c]{@{}l@{}}State attribute typed \\ by deferred event type\end{tabular}        & Not shown in this paper        \\ \hline
		Transition guard                                                            & Method returning a boolean                                                                                           & The valid method at lines 58-59 as the transition guard at line 49                              
	\end{tabular}
\end{table}



\input{sections/oldsynchronization}
\input{sections/synchronizationexample}
