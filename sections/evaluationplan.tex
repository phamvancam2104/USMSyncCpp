\section{Evaluation plan}
\label{sec:evaluationplan}
In order to evaluate RAOES, we plan to conduct experiments focusing on different aspects. 
Our research questions are as followings:
\begin{description}[\footnotesize]
	\item[\tb{RQ1}] A state machine \ttt{sm} is used for generating the front-end code. The latter is reversed engineered to produce another state machine \ttt{sm'}. Are \ttt{sm} and \ttt{sm'} identical? In other words: whether the front-end code generated from USMs model can be used for reconstructing the original model. This question is related to the \ti{GETPUT} law defined in \cite{foster_combinators_2007}.
	
	\item[\tb{RQ2}] The back-end code is used for compilation. 
	Does the runtime execution of the back-end code is semantic-conformant to Precise Semantics for UML State Machines (PSSM)? 
	
	\item[\tb{RQ3}] Runtime performance and memory usage is undoubtedly critical in real-time and embedded systems. Particularly, in event-driven systems, the performance is measured by event processing speed. Does code generated by the presented approach outperform existing approaches and use less memory?
\end{description} 