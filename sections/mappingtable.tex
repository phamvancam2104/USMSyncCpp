\begin{comment}
\begin{table}[]
	\centering
	\caption{My caption}
	\label{my-label}
	\begin{tabular}{lllll}
		UML                  & XGC                      &  & OO                             & C++                 \\ \cline{1-2} \cline{4-5} 
		Class component      & Class                    &  & Class                          & Class               \\ \cline{1-2} \cline{4-5} 
		Part                 & Part                     &  & Composition attribute          & Attribute           \\ \cline{1-2} \cline{4-5} 
		Port  (data/control) & Port                     &  & Attribute                      & Reference Attribute \\ \cline{1-2} \cline{4-5} 
		Many ports           & Multiple-port            &  & Multiple interface realization & --                  \\ \cline{1-2} \cline{4-5} 
		Connector            & Binding (static+dynamic) &  & --                             & Methods             \\ \cline{1-2} \cline{4-5} 
		Interface            & Class/Interface          &  & Interface                      & Class               \\ \cline{1-2} \cline{4-5} 
		Signal               & Class                    &  & Class                          & Class/Struct        \\ \cline{1-2} \cline{4-5} 
		State machine        & state\_machine           &  & --                             & --                  \\ \cline{1-2} \cline{4-5} 
		State                & state                    &  & --                             & --                  \\ \cline{1-2} \cline{4-5} 
		Region               & region                   &  & --                             & --                  \\ \cline{1-2} \cline{4-5} 
		CallEvent            & call\_event              &  & --                             & --                  \\ \cline{1-2} \cline{4-5} 
		TimeEvent            & time\_event              &  & --                             & --                  \\ \cline{1-2} \cline{4-5} 
		ChangeEvent          & change\_event            &  & --                             & --                  \\ \cline{1-2} \cline{4-5} 
		SignalEvent          & signal\_event            &  & --                             & --                  \\ \cline{1-2} \cline{4-5} 
		Any                  & any                      &  & --                             & --                  \\ \cline{1-2} \cline{4-5} 
		Pseudo state         & pseudo\_state            &  & --                             & --                  \\ \cline{1-2} \cline{4-5} 
		Action/Effect        & Method                   &  & Method                         & Method             
	\end{tabular}
\end{table}
\end{comment}

\begin{table}[]
	\centering
	\caption{Mapping between UML and Examples of Extended Language (1)}
	\label{table:mapping}
	\begin{tabular}{p{1.6cm}p{2.5cm}p{3.7cm}}
		UML                                                                      & Extended Language                                                                                & Code example in Fig. \ref{fig:approachexample}                                                                               \\ \hline
		\begin{tabular}[c]{@{}l@{}}Port requiring \\ an interface \ti{I}\end{tabular} & \begin{tabular}[c]{@{}l@{}}Attribute typed \\ by \ti{RequiredPort}\textless I\textgreater\end{tabular} & \begin{tabular}[c]{@{}l@{}}Ports \ti{pPush} and \ti{pPull} at lines\\ 19 and 22\end{tabular}         \\ \hline
		\begin{tabular}[c]{@{}l@{}}Port providing \\ an interface \ti{I}\end{tabular} & \begin{tabular}[c]{@{}l@{}}Attribute typed\\ by \ti{ProvidedPort}\textless I\textgreater\end{tabular}  & \begin{tabular}[c]{@{}l@{}}Ports \ti{pPush} and \ti{pPull} at \\ lines 26-27\end{tabular}            \\ \hline
		\begin{tabular}[c]{@{}l@{}}Bidirectional \\ port providing \\ \ti{R} and \ti{P} \end{tabular} & \ti{BidirectionalPort<R,P>} & Not shown in this paper
		\\ \hline
		Connector                                                                & Binding                                                                                        & Lines 7-8                                                                                  \\ \hline
		State Machine                                                            & \ti{StateMachine}                                                                                     & \begin{tabular}[c]{@{}l@{}}The FIFO state machine at \\ lines 31-51\end{tabular}           \\ \hline
		State                                                                    & \ti{State/InitialState}                                                                               & \begin{tabular}[c]{@{}l@{}}State \ti{SignalChecking} at \\ lines 33-36\end{tabular}             \\ \hline
		Region                                                                   & \ti{Region}                                                                                           & Not shown in this paper                                                                                 \\ \hline
		Pseudo state                                                             & \begin{tabular}[c]{@{}l@{}}Attribute typed \\ by pseudo type\end{tabular}                        & \begin{tabular}[c]{@{}l@{}}The \ti{dataChoice} pseudo state \\ at line 42\end{tabular}          \\ \hline
		Action/Effect                                                            & Method                                                                                           & Methods at lines 52-57       \\ \hline
		Transitions                                                           & Transition table                                                                                           & Transition table at lines 44-50	\\ \hline
		Event                                                            & Event                                                                                           & The call event at line 43       \\ \hline           
		Deferred Event	& \begin{tabular}[c]{@{}l@{}}State attribute typed \\ by deferred event type\end{tabular}        & Not shown in this paper        \\ \hline
		Transition guard                                                            & Method returning a boolean                                                                                           & The valid method at lines 58-59 as the transition guard at line 49                              
	\end{tabular}
\end{table}

