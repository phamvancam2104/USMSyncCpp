\section{RAOES's language syntax}
\label{sec:syntax}
This section presents how USMs can be represented in the front-end language of RAOES.
A USM in RAOES is defined inside an active class (C++ class).
It consists of three parts: topology, event definition, and transition table.
The followings describe the details of these parts.
\subsection{Topology}
A topology describes how the hierarchy of a USM is organized.
As the example in Listing \ref{lst:front-end1}, the root of the topology is define by \ttt{state\_machine}.
Other elements such as \ttt{region}, \ttt{state}, and \ttt{pseudo state} are defined as sub-elements.
The followings give the syntax of some elements and the semantics of RAOES mapped to the well-defined semantics in the UML specification \cite{OMG2015}.

\subsubsection{State and regions} ~\\
\tb{Syntax:}
\begin{itemize}[\footnotesize]
\item \ti{state machine} $\rightarrow$ \tf{'state\_machine('name') \{vertices\};'} 

\item \ti{state} $\rightarrow$ \tf{'state ('name, ent, ex, doAct');'} 

\item \ti{state} $\rightarrow$ \tf{'initial\_state ('name, ent, ex, doAct, effect');'}

\item \ti{concurrent state} $\rightarrow$ \tf{'state ('name, ent, ex, doAct') \{' regions '\};'} 

\item \ti{regions} $\rightarrow$ \tf{region; regions}

\item \ti{region} $\rightarrow$ \tf{'region (' name ') {' vertices '};'}

\item \ti{deferred event} $\rightarrow$ \tf{'defer(' eventId ');'}
\end{itemize}

\noindent
\tb{Semantics:}
\begin{itemize}[\footnotesize]
	\item \ttt{name}: the unique identifier of a state machine, a state or a region.
	
	\item \ttt{ent/ex/doAct}: The name of the entry/exit/doActivity action method associated with the state. 
	These methods are implemented in the active class and have no parameter.
	If a state does not have an entry/exit action, \ttt{ent/ex} becomes \ttt{NULL}.
	
	\item \ttt{initial\_state}: A state is defined as an initial state, which has an incoming transition outgoing from a pseudo initial state within the same region or composite state. 
	
	\item \ttt{effect}: For initial state, this is the transition effect associated with the initial transition.
	If the latter does not have an effect, \ttt{effect} is specified as \ttt{NULL}.
	
	\item \ttt{concurrent state}: The representation of a concurrent state. 
	The latter is composed of a set of regions.
	Each region contains a set of vertices, which for each is either a state or a pseudo state.
	
	\item \ttt{eventId}: The identifier of a defined event (see \ref{subsec:events}), which is deferred by the corresponding state. 
\end{itemize}

\noindent
\tb{Example:}
Lines 2-23 in Listing \ref{lst:front-end1} (a) represents the USM example in Fig. \ref{fig:illustration} (a) and its vertices.
\ttt{S11} is defined as the initial state of the USM, and has \ttt{entry} and \ttt{exit} actions bound to the methods \ttt{S11\_entry} (lines 21-23) and \ttt{S11\_exit} (not shown) implemented in the active class \ttt{System}.
Similarly, the state \ttt{S2} and its sub-vertexes such as \ttt{Enp1}, \ttt{h1}, and \ttt{S21} are defined hierarchically (lines 6-8) with the associated actions.

\subsubsection{Pseudo state} ~\\
\tb{Syntax:}
\begin{itemize}[\footnotesize]
\item \ti{entry point} $\rightarrow$ \tf{'entry\_point ('name');'} 

\item \ti{exit point} $\rightarrow$ \tf{'exit\_point ('name');'}

\item \ti{initial} $\rightarrow$ \tf{'initial ('name');'}

\item \ti{final} $\rightarrow$ \tf{'final\_state ('name');'}

\item \ti{join} $\rightarrow$ \tf{'join ('name');'}

\item \ti{fork} $\rightarrow$ \tf{'fork ('name');'}

\item \ti{choice} $\rightarrow$ \tf{'choice ('name');'}

\item \ti{junction} $\rightarrow$ \tf{'junction ('name');'}

\item \ti{shallow history} $\rightarrow$ \tf{'shallow\_history ('name');'}

\item \ti{deep history} $\rightarrow$ \tf{'deep\_history ('name');'}
\end{itemize}

\noindent
\tb{Example:}

\ttt{entry\_point(Enp1);} and \ttt{shallow\_history(h1);} in Listing \ref{lst:front-end1} represent an entry point and a shallow history with their respective name.  

\subsection{Events}
\label{subsec:events}

Events represent all USM events a USM can react.
As defined in Section \ref{sec:background}, there are four USM event types: \ttt{CallEvent}, \ttt{TimeEvent}, \ttt{SignalEvent}, \ttt{ChangeEvent}. %The syntax of each of these events is as followings.

\noindent
\tb{Syntax:}
\begin{itemize}[\footnotesize]
	\item \ttt{CallEvent} $\rightarrow$ \tf{'call\_event'} \tf{'('name, op');'}
	
	\item \ttt{TimeEvent} $\rightarrow$ \tf{'time\_event'} \tf{'('name, dur');'}
	
	\item \ttt{SignalEvent} $\rightarrow$ \tf{'signal\_event'} \tf{'('name, sig');'}
	
	\item \ttt{ChangeEvent} $\rightarrow$ \tf{'change\_event'} \tf{'('name, expr');'}
	
	\item \ttt{SimpleEvent} $\rightarrow$ \tf{'simple\_event'} \tf{'('name');'}
	
	\item \ttt{Any} $\rightarrow$ \tf{'any\_event'} \tf{'('name');'}
\end{itemize}

\noindent
\tb{Semantics:}
Essentially, each field in the syntax carries known semantics defined in the UML specification and Section \ref{sec:background}:
\begin{description}[\footnotesize]
	\item[\ttt{name}] The unique identifier for an event.
	
	\item[\ttt{op}] The name of the operation associated with a \ttt{CallEvent} and implemented in the active class. 
	
	\item[\ttt{dur}] The duration associated with a \ttt{TimeEvent} and specified as millisecond.
	
	\item[\ttt{sig}] The name of the signal associated with a \ttt{SignalEvent}.
	
	\item[\ttt{expr}] The expression associated with a \ttt{ChangeEvent}. This expression is periodically evaluated to check whether its boolean value is changed.
\end{description}

\ttt{SimpleEvent} is a specialized \ttt{SignalEvent} without specifying an explicit signal.
It is not explicitly standardized by UML but provided by tools such as QM \cite{qm} for practical reasons. 

\noindent
\tb{Example:}

\begin{itemize}[\footnotesize]
\item \ttt{call\_event(CE1, method1)}: A \ttt{CallEvent} occurs if the method \ttt{method1} in the active class is called.

\item \ttt{signal\_event(SE, Sig)}: A \ttt{SignalEvent} occurs if an instance of \ttt{Sig} is sent to the active class using its provided method \ttt{sendSig}.

\item \ttt{time\_event(TE5ms, 5)}: A \ttt{TimeEvent} occurs after 5 millisecond from the moment the timer starts by entering some state.
\end{itemize}

\subsection{Transitions}
As previously defined, there are three kinds of transitions: \ttt{external}, \ttt{local}, and \ttt{internal}.

\noindent
\tb{Syntax:}
\begin{itemize}[\footnotesize]
%\noindent
\item \ttt{external} $\rightarrow$ \tf{'transition'} \tf{'('src, tgt, guard, evt, eff');'}

%\noindent
\item \ttt{local}$\rightarrow$\tf{'local\_transition}\tf{('src,tgt,guard,evt,eff');'}

%\noindent
\item \ttt{internal} $\rightarrow$ \tf{'int\_transition}\tf{('src, guard, evt, eff');'}
\end{itemize}

\noindent
\tb{Semantics:}
\begin{description}[\footnotesize]
	\item[\ttt{src}] The name of the source vertex of the transition. 
	This name must be defined in the topology.
	
	\item[\ttt{tgt}] The name of the target vertex of the transition. 
	
	\item[\ttt{guard}] A boolean expression representing the transition's guard. If the transition is not guarded, \ttt{guard} is NULL.
	
	\item[\ttt{evt}] The name of the event triggering the transition. 
	\tf{evt} must be one of the defined events. 
	If the transition is not associated with any event, \ttt{evt} is NULL.
	
	\item[\ttt{eff}] The name of the method, which defines the effect of the transition.
	The method is implemented in the active class.
	If \ttt{evt} is a \ttt{SignalEvent}, the method has an input parameter typed as the signal associated with the event.
	If \ttt{evt} is a \ttt{CallEvent}, the method has the same parameters as the operation associated with \ttt{evt}.
	If the transition has no effect, \ttt{eff} becomes NULL.
\end{description}

\noindent
\tb{Example:}
\begin{itemize}[\footnotesize]
\item \ttt{transition(S1,S2,guard1,CE1,effect1)}: A transition from \ttt{S1} to \ttt{S2} which is fired if there is an appeal to the method \ttt{method1} and the value of \ttt{guard1} is true.
\ttt{method1} is associated with the \ttt{CallEvent CE1} as the example above.
Furthermore, an action \ttt{effect1} is executed when the transition fires.
Listing \ref{lst:effect-segment} shows how to write \ttt{effect1} (lines 4-6) corresponding to the signature of \ttt{method1} (line 1-3) following the above semantics.

\item \ttt{transition(S11,S2,NULL,SE,effect2)}: A transition, triggered by the signal event \ttt{SE}, executing \ttt{effect2}.
\ttt{effect2} in Listing \ref{lst:effect-segment} has a parameter typed by signal \ttt{Sig} associated with the event \ttt{SE}.
\end{itemize}

\begin{minipage}{1.05\columnwidth}
\begin{lstlisting}[language=C++, caption=A segment of C++ front-end code, label=lst:effect-segment,frame=f]
void method1(int p1, int p2) {
	//method1 body
}
void effect1(int p1, int p2) {
	//effect1 body
}
void effect2(Sig& s) {
	//effect2 body
}
\end{lstlisting}
\end{minipage}