\subsection{Code generation patterns and tools}
Tools such as IBM Rhapsody \cite{ibm_rhapsody}, Enterprise Architect \cite{EA}, Papyrus-RT \cite{possepapyrusrt}, and Sinelabore \cite{sinelabore} support only the structure RTE for UML class diagram concepts and code generation from UML State Machines.
Techniques for generating code from USM such as SWITCH/IF, state table \cite{Douglass1999} and state pattern \cite{Shalyto2006,niaz_mapping_2004} are proposed. 
A systematic review of code generation approaches is presented in \cite{Domnguez2012}.
%Switch/if %is the most intuitive technique implementing a "flat" state machine \cite{Booch1998}.
%and state table approach \cite{Douglass1999} are intuitive techniques. 
%Some approaches such as state pattern \cite{Shalyto2006,niaz_mapping_2004} produce object-oriented code.
% to implement flat state machines. Each state is represented as a class and each event as a method. %The event is processed by a delegation from the context class to sub-states. 
%Separation of states in classes makes the code more readable and maintainable. %Unfortunately, this technique only supports flat state machines. 
%This pattern is extended in \cite{niaz_mapping_2004} to support hierarchical-concurrent USMs. 
%Recently, a double-dispatch (DD) pattern presented in \cite{spinke_object-oriented_2013} extends \cite{niaz_mapping_2004} to support maintainability. %by as a new technique to implement state machines. 
%representing states and events as classes, and transitions as methods. 
However, only a subset of USM features is supported and generated code is not efficient, %of class explosion and uses dynamic memory allocation, 
which cannot be used for embedded systems \cite{spinke_object-oriented_2013}.
%Our source-to-source transformation combines SWITCH/IF the pattern in \cite{niaz_mapping_2004} to produce small footprint and preserve state hierarchy.
%Furthermore, 
RAOES offers code generation for all USM concepts. %including states, pseudo states, transitions, and events.
Therefore, users are free and flexible to create there USM conforming to UML without restrictions.