\section{RAOES as a Component-based language}
We consider the \ttt{component-port-connector} architecture for the representation of system architecture.
Table \ref{table:mapping} shows a mapping from UML component diagram concepts and corresponding elements in DCL.
The following sections describe the DCL elements.


\subsection{Components and parts}
A component is one of the constituents of a system implementation.
It encapsulates its content, which can be composed a set of sub-components or a class in object-oriented programming (OOP) languages.
Each sub-component is considered as a part of the component.

Listing \ref{lst:system} shows the description of the example in Fig. \ref{fig:cbseexample}.
\ttt{System} is implemented as an OO class and consists of three parts: a producer \ttt{p}, a consumer \ttt{c}, and a first-in first-out (FIFO) communication channel \ttt{fifo}.


\begin{minipage}{0.95\columnwidth}
	\lstinputlisting[language=C++, caption=Front-end code generated from the System in Fig. \ref{fig:cbseexample}, label=lst:system,frame=f]{code/system.cpp}
\end{minipage}

The behavior of a component is described by a state machine written in RAOES. 

A component interacts with other components via interaction points, namely, ports, which are described in the following sub-section.

\subsection{Ports and binding}
A port can be a data/message port or a method call/control port.
While a data port of a component exchanges data with other components,
a control port uses interfaces and method calls for interacting with other components. 
A port provides/requires one or multiple data types, for data port, and/or one or multiple interfaces, for control port.
Note that this port definition is different from UML's, in which a port can provide/require one interface.

Listing \ref{lst:systemparts} displays the code segment generated for the example.
The producer and the consumer require the interfaces \ttt{IPush} and \ttt{IPull} to interact with the fifo via their respective ports \ttt{pIPush} and \ttt{pIPull}.
The interfaces contain method declarations, which are implemented by a certain component, the fifo in this case.

\begin{minipage}{0.95\columnwidth}
	\lstinputlisting[language=C++, caption={DCL code generated Producer, Consumer, and FIFO}, label=lst:systemparts,frame=f]{code/systemparts.cpp}
\end{minipage}

\ttt{Binding}s are used for connecting ports of different components for configuration of component communication in a system.
A binding is created by an invocation to the function \ttt{bindPorts}, which takes as input two ports as shown in Fig. \ref{fig:cbseexample}.
A binding is similar to the concept of UML connectors.
It can be a delegate or an assemble binding.
A delegate binding connects two ports of components that have a hierarchical relationship while ports in an assembly binding do not.

\vskip 0.03in
\noindent
\tb{Static and dynamic component creation and bindings:}
Each composite component has a unique configuration.
It is used for creating static bindings during system initialization by putting all of bindings inside the system configuration as in Listing \ref{lst:system}.
However, components and bindings can be created dynamically by using \ttt{instance creation expression (new)} of OOPLs and the \ttt{bindPorts} function, respectively, so that the topology of system's components can be changed at runtime. 
This feature is useful in adaptive systems.

For example, the communication component \ttt{fifo} in Fig. \ref{fig:cbseexample} can be replaced by a priority queue by instantiating \ttt{prioQueue} and binding the producer to it as in Listing \ref{lst:dynamic}.

\begin{minipage}{0.95\columnwidth}
	\lstinputlisting[language=C++, caption={Dynamic component and binding}, label=lst:dynamic,frame=f]{code/dynamic.cpp}
\end{minipage}

\vskip 0.03in
\noindent
\tb{Usage of provided interfaces:} A port with required interfaces bound to a port with provided interfaces through the methods (services) defined in the interfaces in an object-oriented way.
For example, the \ttt{p} producer can push data items to \ttt{fifo} by calling the \ttt{push} method via the \ttt{pPush}: \ttt{p.pPush.push(item)}. 

The \ttt{FIFO} component contains two provided ports \ttt{pPush} and \ttt{pPull}.
However, using the concept of port with multiple required interfaces, we can re-write \ttt{FIFO} and the configuration in \ttt{System} as followings:

\begin{minipage}{0.95\columnwidth}
	\lstinputlisting[language=C++, caption={FIFO using a port with multiple required interfaces}, label=lst:multiple-port,frame=f]{code/multiple-port.cpp}
\end{minipage} 

Data ports can be used instead of control ports in the example.
Data items flow from the port providing to the port requiring the data.
Data ports are defined to support the explicit understanding of "physical" system data flow. 
Fig \ref{fig:dataportexample} shows the example of using data ports.
The \ttt{p} producer provides data items to \ttt{fifo} through their respective ports \ttt{pProvideData} and \ttt{pRequireData}.
The code for the data ports is shown in Listing \ref{lst:dataport}.

In system implementation, the producer provides data items to the fifo via the port \ttt{pProvideData} by calling \ttt{pProvideData.send(item)} as object-oriented way.




\subsection{Consistency of ports}
This section presents the constraints for ports and bindings.
The constraints for a port \ttt{p} of a component \ttt{Comp} are stated as followings:

%\begin{itemize}
	%\item An intermediate language CDL - Component-based Description Language is proposed for programmers
	%\item CDL is generated from model or can be created by programmers
	%\item CDL is used for whole system implementation, not just architecture description
	%\item CDL adds more constructs to OO languages:
	%\begin{itemize}
	%	\item UML State Machine constructs: state, transition, event, pseudo state, action
		
	%	\item Component constructs: component, part, port, connector
		
	%	\item Reuse maximally existing constructs in OO languages such as class, attribute
	%\end{itemize}
	
	%\item CDL is synchronized with models
%\end{itemize}




\begin{itemize}
	%\item A port can provide or require multiple interfaces/data
	
	%\item For port p of component C
	
	%\begin{itemize}
	\item If \ttt{p} provides an interface \ttt{I} => \ttt{Comp} or one of \ttt{Comp}'s sub-com must implement I. 
	
	\item If \ttt{p} requires a data type/signal \ttt{D} => A realization of an interface named \ttt{ISendD}, which contains only one method \ttt{sendD}, is generated for \ttt{Comp}.
	
	%\item If p provides a data signal Sig => C sends signal via p by calling p->sendSig
	%\item If p uses a data signal Sig => C/C's sub-component implements ISig interface, which contains only a method sendSig(Sig\& sig);
	%\end{itemize}
	
\end{itemize}


Two ports \ttt{p1} and \ttt{p2} match for binding if one of the following conditions is met:
\begin{itemize}
	\item Assemble: All required interface and data types of p1 (p2) match with all provided interfaces and data types, respectively, of p2 (p1).
	
	\item Delegation: All required interfaces and data types of p1 (p2) match with all required interfaces and data types, respectively, of p2 (p1).
\end{itemize}

\noindent
\tb{Match:} an interface/data type A matches with an interface/data type B if A = B or A inherits from B.

