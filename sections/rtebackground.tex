\subsection{Model-Driven Round-trip Engineering}
\label{subsec:mdrtebackground}

This section defines the actors in software development process who will use our model-code RTE technique to collaborate during development.
%Then we define the main capabilities, as use-cases, expected from a generic IDE used by these actors.
Some basic concepts related to the actors and use-cases, which will be offered by RAOES, are also defined in this section.


%\subsection{Collaborating actors and development artifacts}

%In this paper we propose a methodological model-code synchronization pattern for collaboration between software
%architects and programmers.

%First, we introduce the concepts of \textit{development artifact} and \textit{baseline artifact}.

%\begin{definition}[Development artifact]
%	A development artifact is an artifact, as defined in \cite{omg_software_2008},
%	that can be used for the full implementation of the system.
%\end{definition}

%For example a system can be entirely implemented as code.
%Implementation code is a development artifact, so may model.
%It is then not only documentation of specification
%but part of the implementation.
%For example a model can be used for implementation by generating code from the model, and compiling the code without the need to edit or complete the code.
%In our work, we assume that model and code are both development artifacts.
%A development artifact may be the baseline artifact, defined in this paper as follows:

\begin{comment}

\begin{definition}[Baseline artifact]
	A baseline artifact is one which may be edited manually.
	All other artifacts are produced from the baseline artifact
	through some process, and only through a process. Manual edition
	of artifacts other than the baseline artifact is forbidden.
\end{definition}
\end{comment}

Two primary actors, called \ttt{model-driven developer}
and \ttt{code-driven developer}, are introduced.
%The main difference between them
%is what they consider as the baseline artifact.

\begin{definition}
	A model-driven developer (MDD) is an actor who uses the model as the main working artifact. Code-driven developer (CDD) is an actor who uses the code as the main working artifact.
\end{definition}

%In other words, for the model-driven developer only the model should be edited manually. 
The code, produced from the model automatically, is consistent with the model.
A software architect is a kind of the model-driven developer
who edits the model to specify the system architecture.
%An architect presumes that the reference for the architecture
%of the system should be specified as a model.



A programmer is a specialization of the code-driven developer.
Indeed, programmers may modify our C++ front-end code, such as editing methods, attributes or state machines textually.
%The code is then the main reference for the implementation of methods.

%There are some use-cases for manual edition of artifacts. The \texttt{Edit Artifact} use-case
%implies that the IDE must have some tool to let the developer manually edit an artifact.
%The \texttt{Edit Model} and \texttt{Edit Code} use-cases are specializations of the \texttt{Edit Artifact}
%use-case where the artifact is the model or code.

%There are also some use-cases related to the synchronization of artifacts. The \texttt{Synchronize Artifact} use-case (1) compares two artifacts, (2) updates each with editions made
%in the related artifact, and (3) reconciles conflicts when appropriate. The \texttt{Synchronize Model} and \texttt{Synchronize Code}
%use-cases are specializations where, respectively, the model or the code are the artifacts being synchronized.

\texttt{Generate Code} is a use-case related to forward engineering.
It is the production of C++ code from a model.
The developer can either use \texttt{Generate Code (Batch)} or \texttt{Generate Code (Incremental)}.
%\end{comment}

\begin{definition}[Batch code generation \cite{Giese2006}]
	Batch code generation is a process of generating code
	from a model, from scratch.
	Any existing code is overwritten by the newly generated code.
\end{definition}

%\ttt{Incremental code generation} is a specialization of \ttt{incremental model transformation}, which
%does not generate the whole target model from scratch but only updates the target model by
%propagating editions made to the source model.

%\texttt{Incremental code generation (ICG)} \ti{$gen_{inc}$ is a process of taking as input a changed model m and an existing executable code to make the code synchronized with the changed model: $gen_{inc}(m, c) = c'$. Non-conflicted changes at the code side are kept intact the synchronization. ICG is also defined as a process of taking model changes ch and an existing code c: $gen_{inc}(ch, c) = c'$}.

%Derived from the definition of incremental model transformation, 
%Incremental code generation
%is defined in this paper as follows:

\begin{definition}[Incremental code generation]
	Incremental code generation is the process
	of taking as input an edited model and existing code to update the code by propagating
	editions in the model to the code.
\end{definition}

%\begin{comment}
\texttt{Reverse Code} is related to reverse engineering.
\texttt{Reverse Code} is the production of a model, in a modeling language such as UML, from code, written in a programming language.
The developer can either use \texttt{Reverse Code (Batch)} or \texttt{Reverse Code (Incremental)}, which are defined in this paper as follows:
%\end{comment}

\begin{definition}[Batch reverse engineering]
	Batch reverse engineering is a process of producing a model from code, from scratch.
	The existing model is overwritten by the newly produced model.
\end{definition}

\begin{definition}[Incremental reverse engineering]
	Incremental reverse engineering is the process of taking as
	input a edited code, and an existing model, and then updating the model by propagating
	editions in the code to the model.
\end{definition}

%For readability, in this paper we will sometimes designate batch and incremental as modes
%of code generation/reverse; e.g. we say that we generate code in batch mode from a model.

%The use-cases are generic. They do not depend
%on any particular approach or tool. Therefore the software developers
%can choose the approach or tool that suits better his/her
%development preferences best.

In Section \ref{sec:collaboration}, the use-cases are integrated into
our process, which covers model-code synchronization and is detailed in Section \ref{sec:collaboration}.
%The scenarios correspond to behaviors performed by both kinds of actors,
%i.e. model-driven developers and code-driven developers.

%Model-driven engineering has been established as a potential approach to gain software quality and productivity \cite{Mussbacher2014}. 