\begin{abstract}
Event-driven architecture is an useful way to design memory-constrained embedded systems.
Unified Modeling Language State Machine and its visualization are a powerful means to the modeling of the logical behavior of such architecture.   
Model Driven Engineering generates executable code from state machines. 
The generated code can then be modified by programmers.
Round-trip engineering is a technique used to propagate changes made to code to the original model.
However, existing round-trip engineering tools and approaches mainly focus on structural parts of the system model such as those available from UML class diagrams.

In this paper, we tackle the problem of collaboration between software architects and programmers in developing event-driven embedded systems using UML State Machine to describe the behavior.
We propose a round-trip engineering and synchronization of model and C++ code, with which the software architects and programmers can freely switch between model and code to be efficient in their preferred practice.
\end{abstract}