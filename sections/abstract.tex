\begin{abstract}
%Event-driven architecture is an useful way to design memory-constrained embedded systems.
Architecting a software system has become one of the most important tasks during development.
UML State Machine and Composite Structure are efficient to design reactive system architectures.
%A UML State Machine and its visualizations are efficient to model the logical behavior of event-driven embedded systems.   
In Model Driven Engineering (MDE), code can be automatically generated from the models. 
%To achieve it, models must contain very detailed information.

However, current UML tools and approaches are not sufficient to describe and generate the fine-grained behavior. 
Hence, these tools allow manually putting programming code as blocks of text embedded within the model for the fine-grained behavior.%Current MDE tools therefore allow to manually put action code directly within the architecture model by limited editors,
This manual coding practice in a limited editor loses programming flexibility and might produce errors during compilation of the generated code.
In this case, programmers tend to directly modify the code using familiar integrated development environments.
%current UML tools are hardly to generate fully operational code but produce only skeleton code which is then fine-tuned by programmers.
The modifications in code, which may violate the architecture correctness, must be synchronized with the model to make architecture and code consistent.
Current approaches cannot handle the synchronization of model and code in case of UML State Machine and Composite Structure elements because there is a significant abstraction gap between architecture and code.

This paper proposes an approach to enable this synchronization. %the architecture model in UML State Machine and Composite Structure elements, and the code.
The approach consists of a bidirectional mapping between code and the architecture model, and a synchronization mechanism, which allows to synchronize concurrent modifications made in model and code.
%The proposed mapping is a means for a synchronization mechanism proposed in our previous work, which allows concurrent modifications made in model and code, and keeps them synchronized. 
%for the co-evolution of these artifacts.
%We propose XSeparation - a set of utilities for enabling the bidirectional traceability and synchronization between architecture model and code.
%To do it, RAOES proposes to generate an intermediate representation, which acts as a bridge to connect the code to the model.
The approach is then evaluated through the development of an embedded software case study - a Lego Car factory.
The evaluation shows that our approach can automatically synchronize model and code, and eventually improve collaboration between software architects and programmers.  
%Furthermore, the quality of generated code is also preserved within our approach.
%We propose an evaluation plan for the approach and expose preliminary experimental results.
\end{abstract}