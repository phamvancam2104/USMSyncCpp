\begin{abstract}
%Event-driven architecture is an useful way to design memory-constrained embedded systems.
UML state machines and composite structure models are efficient to design the behavior and structure of architectures.
%A UML State Machine and its visualizations are efficient to model the logical behavior of event-driven embedded systems.   
In Model Driven Engineering (MDE), code can be automatically generated from the models. 
%To achieve it, models must contain very detailed information.
Nevertheless, %current MDE tools and approaches are not sufficient to describe and generate the fine-grained behavior from models described as UML composite structures with UML state machines. 
%Current MDE tools therefore allow to manually put action code directly within the architecture model by limited editors, in which code errors might be produced during generated code compilation.
%In this case, the programmers tend to directly modify the code in familiar integrated development environments.
current UML tools only produce skeleton code which is then fine-tuned by programmers.
The modifications in code, which may violate the architecture correctness must be synchronized with the model to make architecture and code consistent.
However, current approaches cannot handle the synchronization when there is a significant abstraction gap between architecture and code.
This paper proposes to ease synchronization between model and code, through a bidirectional mapping between code and architecture models 
specified by UML composite structure and state machine.
The proposed mapping is a means for a synchronization mechanism proposed in our previous work, which allows concurrent modifications made in model and code, and keeps them synchronized. 
%for the co-evolution of these artifacts.
%We propose XSeparation - a set of utilities for enabling the bidirectional traceability and synchronization between architecture model and code.
%To do it, RAOES proposes to generate an intermediate representation, which acts as a bridge to connect the code to the model.
We propose an evaluation plan for the approach and expose preliminary experimental results
\end{abstract}